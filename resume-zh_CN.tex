% !TEX TS-program = xelatex
% !TEX encoding = UTF-8 Unicode
% !Mode:: "TeX:UTF-8"

\documentclass{resume}
\usepackage{zh_CN-Adobefonts_external} % Simplified Chinese Support using external fonts (./fonts/zh_CN-Adobe/)
%\usepackage{zh_CN-Adobefonts_internal} % Simplified Chinese Support using system fonts
\usepackage{linespacing_fix} % disable extra space before next section
\usepackage{cite}
\usepackage{graphicx}
\usepackage{tabu}
\usepackage{multirow}

\begin{document}
\pagenumbering{gobble} % suppress displaying page number

% \name{赵\hspace{1em}炳}

% \basicInfo{
%   \email{zhoabingcn@hotmail.com} \textperiodcentered\ 
%   \phone{(+86) 18222917001} \textperiodcentered\ 
%   \linkedin[zhaobing]{https://www.linkedin.com/in/billryan8}}

\basicInfo{
  \begin{tabu}{ c r r }
   \multirow{2}{1in}{\name{赵炳}}\hspace{11em} &  &  \\
    & \email{zhaobingcn@hotmail.com} & \pbar{Java}{0.75} \\
    Java研发工程师& \phone{(+86) 182-2291-7001} & \pbar{Python}{0.5} \\
    数据分析工程师& \github[github.com/zhaobingcn]{https://github.com/zhaobingcn} & \pbar{Scala}{0.5} \\
    % & \github[github.com/billryan]{https://github.com/billryan} & \pbar{Javascript}{0.5}
  \end{tabu}
}
 
\section{\faGraduationCap\  教育背景}
\datedsubsection{\textbf{北京邮电大学},计算机学院,计算机科学与技术}{2015年9月 -- 至今}
\textit{工学硕士}\ ,预计 2018 年 3 月毕业
\datedsubsection{\textbf{天津工业大学},计算机学院, 主修软件工程,辅修英语}{2011年9月 -- 2015年6月}
\textit{工学文学双学士}\ ,成绩排名:2/244

\section{\faCogs\ IT 技能}
% increase linespacing [parsep=0.5ex]
\begin{itemize}[parsep=0.5ex]
  \item 熟悉常用的数据结构,常用算法和设计模式,并且可以运用到实际工作中。
  \item 精通Neo4j,TitanDB等NoSQL图数据库,熟悉社交网络分析技能。
  \item 具备MySQL,MongoDB数据库的开发经验,熟悉数据的检索,查询优化等技能,熟悉Lucene。
  \item 熟悉Web框架Spring,MVC,MyBatis,Spring Data,RESTful以及Velocity,Thymyleaf等前端模板技术,可以熟练的使用Spring Boot,Maven,Gradle等搭建开发环境。
  \item 了解分布式系统,Docker虚拟化技术,可以搭建Spark,Hadoop分布式集群,并且应用到实际项目中去。
\end{itemize}

\section{\faUsers\ 实习/项目经历}
\datedsubsection{\faBank \textbf{中国电子科技集团信息科学研究院} ,\faMapMarker 北京}{2016年4月 -- 2017年2月}
\role{图形大数据项目}{项目研发}
% 大数据项目开发
\vspace{-0.7em}
\begin{onehalfspacing}
\begin{itemize}
  \item 项目以Neo4j作为数据库(负载均衡集群),采用Spark作为计算引擎,以MVC模式为项目框架
  \item 实现了一个以科研资料知识图谱为基础的分析系统,可以实时的获取数据(定时爬取数据导入),并且实时的分析(定时的进行数据分析,以向用户展示最新分析结果)
  \item 千万级别图数据的算法分析可以在秒级别完成(实测PageRank10M点,120M边迭代五次只需5s)
\end{itemize}
\end{onehalfspacing}
\datedsubsection{\faBank \textbf{中国电子科技集团信息科学研究院} ,\faMapMarker 北京}{2016年4月 -- 2017年2月}
\role{基于微服务的社交网络分析系统}{独自开发}
\vspace{-0.7em}
\begin{onehalfspacing}
% 分布式负载均衡科学上网姿势, https://github.com/cyfdecyf/cow
% 实时社交网络分析平台,为了验证技术可行性
\begin{itemize}
  \item 使用Docker和SpringBoot构建微服务,用Neo4j做存储,Spark Graphx做计算,并且接入TwitterAPI
  \item 实现了一个可以分析社交网络中节点重要性的系统,使用TwitterAPI抓取数据,使用Spring Data Neo4j管理得到的数据,构建成网络模型,使用Spark GraphX计算并且返回计算结果到Neo4j
  \item 实现了从社交网络的一个人开始,他的朋友网络之间的关系,影响力等分析
\end{itemize}
\end{onehalfspacing}

% \datedsubsection{\textbf{\faBank\hspace{0.2em} Neo4j官方中文社区}}{2016 年11月 -- 至今}
% 担任Neo4j官方中文社区版主,兼任维护人员(保证网站正常访问,统计流量等),推广Neo4j
% \begin{onehalfspacing}
% % 优雅的 \LaTeX\ 简历模板, https://github.com/billryan/resume
% \begin{itemize}
%   \item 日常维护Neo4j中文社区,推广Neo4j的应用,解答问题
%   \item 在此期间与Neo4j中国合作伙伴微云数聚一起完成了《Neo4j权威指南》的写作(清华大学出版社,5月份出版,已交稿)
% \end{itemize}
% \end{onehalfspacing}

% Reference Test
%\datedsubsection{\textbf{Paper Title\cite{zaharia2012resilient}}}{May. 2015}
%An xxx optimized for xxx\cite{verma2015large}
%\begin{itemize}
%  \item main contribution
%\end{itemize}

\section{\faStar\ 获奖情况}
\datedline{\textit{全国二等奖}, 全国大学生数学建模大赛}{2013 年8 月}
\datedline{\textit{研究生一等奖学金},北京邮电大学}{2015年-2017年}
\datedline{\textit{校级一等奖学金(每年),优秀学生(每年),优秀毕业生},天津工业大学}{2012年-2015年}

\section{\faHeart\ 技术心得}
% increase linespacing [parsep=0.5ex]
\begin{itemize}[parsep=0.5ex]
  \item 个人技术博客\faLink : http://zhaobing.me
  \item GitHub\faLink : https://github.com/zhaobingcn
  \item 语言: 英语 - 本科阶段辅修英语并获得文学学士学位,六级
\end{itemize}

%% Reference
%\newpage
%\bibliographystyle{IEEETran}
%\bibliography{mycite}
\end{document}
